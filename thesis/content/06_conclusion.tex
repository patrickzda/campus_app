\chapter{Conclusion}
\label{cha:conclusion}

This section portrays the features of the developed campus app and discusses their impact on the introductory problems regarding TU Berlin's digital information culture as well as the limited navigation capabilities across the campus. It furthermore provides an overview of possible starting points and aspects for future work. The conclusion thereby constantly focuses on the results of the quantitative evaluations performed in the previous chapter and tries to provide additional resources and ideas for performing a qualitative verification, which is expected to be a natural continuation of the evaluation regarding the campus app's impacts on usability and user experience.

\section{Campus app and its impact on introductory problems}
The two main problems formulated in the introduction lie in the complicated and overwhelming digital information culture of TU Berlin, which distributes campus information across multiple web resources and in the insufficient provision of navigation and orientation possibilities for students and teaching staff of campus Charlottenburg. Both problems are tackled by the two main features of the campus app:

On the one hand, a digital navigation system for routing and localizing on campus Charlottenburg tries to solve the orientation weaknesses of TU Berlin's current manual campus map and the broadly used location-based mobile navigation systems Google and Apple Maps. It offers the ability to calculate the fastest route between different POIs or to navigate the user from the current position to an entity on the campus. This system is thereby designed in a way that adapts to the common needs of students and teachers of TU Berlin by e.g., including campus-specific pathways and buildings for fast and direct navigation.

The most influential comparison point between the campus app's navigation system and already established providers (Google and Apple Maps) are the lengths and walking times of suggested routes. As portrayed in the evaluation benchmark presented at \hyperref[sec:navigation_system_verification]{"Navigation system verification" section}, the campus app's routing system with its underlying street network graph outperforms the two other solutions by 30\% in terms of average route time savings and by 23\% (Google Maps) / 20\% (Apple Maps) in terms of average walking distance saved per route. Based on these figures, the underlying work can be considered a successful improvement of the navigation situation on TU Berlin's main campus and therefore solves the related problem formulated in the introduction.

% Hier die Ergebnisse der Evaluation des Navigationssystem einfügen

On the other hand, an interactive information layer inside the app tries to simplify the handling of the information provided by TU Berlin and Studierendenwerk. This layer presents different information about buildings, canteens, rooms, events and course timetables while trying to enhance the user experience by providing different ways for exploring and searching through the data. Since the major contribution of this system lies in the provision of better usability, direct qualitative analysis is in this case mostly not feasible for evaluation (see section \hyperref[sec:future_work]{"Future work"}).

% When compared to ...

One important element contained in the campus app is a 3-dimensional map of campus Charlottenburg, which serves as a building block for both the navigation system as well as the information layer and can be therefore considered another important feature of the app. Apart from providing the underlying layer for the other features, the campus map also serves as a digital reference environment in which campus Charlottenburg can be explored by the user digitally. A high priority is thereby placed on the usability of the map, with a strong emphasis on the recognizability of buildings, pathways and green areas on the campus and a form factor inspired by the already familiar official campus map of TU Berlin.

To achieve the described set of features, the campus map is powered by the app's geocoding services which connect the geodata to their respective POIs in human-readable format (names and abbreviations of buildings, addresses, \ldots).

To evaluate the quality of the campus map and its underlying geodata, two distinct benchmarks are presented in the \hyperref[sec:campus_map_verification]{"Campus map verification" section}. The provided data shows that, on average, the density of geodata points on the campus present in the developed solution is higher than in the compared mobile map services Google Maps (by factor 2) and Apple Maps (by factor 5). Furthermore, it can be shown that the campus app outperforms both other systems in terms of knowledge of campus-specific identifiers, abbreviations and names of TU Berlin's entities. When compared, both systems could only correctly allocate 21\% (Google Maps) / 2\% (Apple Maps) of the officially used abbreviations and 49 \% (Google Maps) / 17\% (Apple Maps) of the building names on campus. The campus app thereby scored 100\% in both categories respectively.

It can be therefore concluded that the goal of creating a more detailed and adapted campus map of TU Berlin's campus could be achieved during development.

\section{Future work} \label{sec:future_work}
To conclude this thesis, an outlook of ideas and possibilities to continue the campus app project is presented in this section. Special attention is thereby paid to qualitative evaluation of user experience, improvement of the current version of the app as well as to possibilities for work based on data related to the campus app.

Evaluation of user experience: One key prospect and goal during campus app development is the enhancement of the usability of TU Berlin's digital web resources. Special attention is paid to providing TU Berlin's data in an intuitive and visually appealing format for usage on mobile devices. Since it is unfortunately not possible to verify these aspects in the scope of this work, a qualitative usability study could be seen as an important next step. Such evaluation could be performed by selecting a representative group of people studying and working on campus Charlottenburg, which would test and compare the developed app against the online resources of TU Berlin and Studierendenwerk. This could help reveal the strengths and weaknesses of such a system during practical usage and would provide a starting point for further adjustments of the user experience and user interface design work lying behind the campus app.

\newpage

Performance: Looking from an app development perspective, one aspect worth evaluating is the performance of the app. Since no particular focus is set on this topic during this work, several optimizations can be performed for a potential future version. Major key aspects in this context are the improvement of loading times for offline datasets (especially course and CourseEvent data) and optimizations for the 3d rendering capabilities related to the digital campus map.

On the one hand, course and CourseEvent data could be split into several files based on the date/buildings/preferences of the user. This could allow the system to load individual smaller chunks of data when needed instead of copying the complete set into the RAM on startup. Other possibilities could be the use of compression techniques to minimize disk size as well as pre-sorted data to speed up the search processes performed in the app.

On the other hand, the lighting settings as well as the resolution and quality of the models used in the 3d visualization of the campus could be reconsidered. Shapes of buildings could be simplified and a correct choice of triangulated faces could cut the rendering workload for the buildings by half due to backface culling. The impact of shadows, post-processing and other visual effects used in the 3d representation could be evaluated and effects with a low contribution to usability and visual appearance discarded.

Full automation of data collection: Regarding the deployment of the app in a production environment, full automation of the data collection process would be beneficial for the user experience as well as for the maintenance workload. Web crawling systems that currently need manual activation could be deployed on a web server and provide their data online. A fitting download and storage system on the client side could then query the data at the start of each semester and save it locally for intended offline usage. However, integrating the web APIs of the relevant parties would be the most sustainable and efficient solution for providing the data in the long term.

Improvement of features: One major possibility for future work is the further development of existing and new features for the campus app. Examples of such improvements could be the integration of more data providers (e.g., the Congeno event system, job and thesis markets of TU Berlin, \ldots), the integration of a personal timetable with customized courses as well as the implementation of a tracking tool for social contacts on campus or course attendance. It is important to note that a careful selection of new features is key to good usability and that too many complex features could risk a bloating of the app. User tests, surveys across students and teachers and other qualitative evaluations should be therefore the building blocks and initiators of new features. 

Scientific usage of data: One key enabler of the campus app is the underlying data collected and processed in the scope of this thesis. Future work could use this data for analysis and evaluation of campus Charlottenburg as well as the online resources of TU Berlin. Taking a live production release of the app as a premise, it is furthermore possible that several different tracking and data collection systems could be integrated into the app. Scientific research based on such data could be also considered as future work.

% Verbesserung der Performance (Optimierung der Campus-Karte, Verbesserung der Ladezeiten für offline Daten, vor allem die Stundenpläne von Kursen)
% Automatisierung des Systems zur Generation von Stundenplänen und Raumbelegungen (Automatisches Generieren -> Bereitsstellen der Dateien auf einem Server -> System, mit welchem die App die Daten zu Beginn des Semesters herunterlädt und offline für den Rest des Semesters bereitstellen kann)
% Einbeziehen von weiteren öffentlichen Online-Daten der TU Berlin und des Studierendenwerks (ggf. noch weiterer Parteien?)
% Entwicklung weiterer Features
% Wissenschaftliche Analyse von campus-spezifischen Datensätzen, welche mit einem möglichen Release der App generiert werden könnten
% Verwendung der im Rahmen dieser Bachelorarbeit gesammelten Daten zu TU Berlin's Gebäuden, dem Straßennetzwerk der Uni, ...
% Qualitative Tests zum Messen der User-Experience / zum Vergleichen der App und der digitalen Web-Resourcen der TU Berlin