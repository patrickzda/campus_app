\chapter{Related Work}
\label{cha:relatedwork}

\section{GPS}
The global positioning system (GPS) as described in \cite{272176} and \cite{1013999415003} is the modern standard for digital localization on Earth. It is widely used across several different domains including navigation, tracking functionalities, military usage and other location-based applications. At least since the rise of mobile devices equipped with GPS capabilities, the majority of people regularly use GPS within several mobile apps and services.

The basis of GPS technology consists of a synchronized satellite network, which constantly broadcasts status information about the respective position, orbit and time of its members. GPS devices calculate their latitude, longitude and altitude by receiving data from at least four satellites. After data is transmitted, the signal propagation time is used for lateriation, from which the position of the device can be determined. \cite{lateriation} presents the mathematical background for such position determination: Based on the signal runtime, the distance R\textsubscript{i} between each satellite $i$ and the receiver is calculated. With the unknown receiver position $(x, y, z)$ and the known satellite position $(x_{i}, y_{i}, z_{i})$ the corresponding distance can be expressed as follows: $(x_{i} - x)^{2} + (y_{i} - y)^{2} + (z_{i} - z)^{2} = R_{i}.$ This set of (at least 3) equations can be then solved for the receiver position $(x, y, z)$.

Since a correct measurement of signal runtime is essential (errors of 1 ms result in uncertainties of around 300 km \cite{1013999415003}) and consumer-friendly GPS devices usually only contain a simple quartz-based clock, error estimation and correction systems are needed to asses the exact time.

Depending on the environment and scenario in which GPS is used, several optimizations and improvements can be made. Since this thesis implements a navigation system for mobile devices, smartphone-focused implementation and usage techniques are particularly interesting.

One important example of such an improvement is the Google fused location provider API. This API is specifically designed for usage on mobile devices and improves the efficiency and accuracy of location retrieval by combining several internal smartphone sensors such as GPS and WiFi \cite{fused_location_api}.

\section{OSM mapping service}
OpenStreetMap (OSM) is a community-driven open-source mapping service that provides annotated digital maps for most of the countries in the world. It was initiated in 2004 and consists of a web application \cite{openstreetmap_website} that hosts a static map as well as the Overpass Turbo API \cite{openstreetmap_overpass_turbo} for data retrieval.

One problem arising from the community-based approach of OpenStreetMap is the fact that it lacks quality assurance \cite{quality_of_openstreetmap}. OSM quality varies depending on the country and region \cite{quality_of_openstreetmap}: While rural areas often lack information, urban locations are often precisely represented and contain information comparable to data provided by the respective government of the selected area or commercial companies.

\cite{quality_of_openstreetmap}, with its implementation of the web app `Is OSM up-to-date?´ provides a systematic approach for measuring the OSM quality of an area. It presents how recently certain areas have been edited by the community and infers from that the need for revision on the map. In the case of TU Berlin's campus in Charlottenburg, the available data is well-maintained \cite{is_osm_up_to_date} and can be used as a starting point for this thesis.

\section{Techniques used in navigation systems for mobile devices}
Navigation systems for mobile devices are nowadays a crucial part of the landscape of available navigation solutions. Popular mobile apps, such as Google \cite{google_maps_website} or Apple Maps \cite{apple_maps_website}, often provide users the ability to navigate, locate and explore the world around them. The core features of those systems and important aspects for this thesis are:

\subsection{Routing}
Routing through a network of streets and places is an important key functionality of every digital navigation system. The most popular techniques for routing all use an underlying graph representation of the street network \cite{google_maps}. With this technique, streets are represented as weighted nodes, which model the underlying cost of moving between different locations.

One of the most popular algorithms to determine the fastest way between two nodes is Dijkstra's Algorithm. It greedily computes the shortest paths in the whole network and always returns the optimal solution (the fastest path). Despite its popularity and effectiveness, the algorithm's runtime scales poorly for huge graphs, e.g., the underlying data of Google Maps \cite{google_maps}. This is the reason why most of the algorithms used for routing in a navigation environment use different heuristics to speed up calculation. These algorithms usually trade off optimal route calculation for increased execution speed \cite{routing_algorithms}. The following list presents an overview of the basic Dijkstra implementation and other commonly used routing algorithms \cite{routing_algorithms}.

Dijkstra: This algorithm begins at a pre-defined start node and calculates the movement cost to all directly reachable nodes. It enqueues all visited nodes and repeats this process for the first node in the queue. Processed nodes get dequeued and costs for movement are updated when a shorter path is found. By calculating all possible paths in the graph, the algorithm is guaranteed to find the optimal solution. The main disadvantage of this approach is its runtime for huge graphs. Use cases consist of networks with small graph representations, several internet routing protocols and educational purposes.

A*: A* is a generalized version of the Dijkstra algorithm. The difference can be found in the cost function for movement between two nodes: Instead of only using the edge weight, A* applies a heuristic to the visited nodes. In a routing scenario for navigation purposes, the heuristic usually consists of the airline between the current node and the destination and improves calculation speed by providing the algorithm an indication of nodes that can be prioritized in the search. A* also always finds the optimal solution, while being faster than Dijkstra in most cases. Use cases are routing in bigger networks and educational purposes.

Genetic algorithms: Genetic algorithms describe a class of techniques for routing based on evolutionary processes. These algorithms start by choosing a random path and treating it like a gene. Principles like genetic crossover and random mutations are applied to it and the paths with the least movement cost are used for creating new populations of genes. This evolutionary process often converges to an acceptable solution in an adequate time. Its biggest disadvantage is the fact that it does not necessarily provide an optimal solution. Use cases are large graphs and scenarios that do not need an optimal solution.

%One example of them is the A* algorithm: It works similarly to Dijkstra's algorithm but applies a heuristic to every node in the graph. By prioritizing the promising paths in the graph, A* usually finds the optimal routing solution faster than the basic Dijkstra implementation \cite{google_maps}.

%Since this thesis only focuses on TU Berlin's campus Charlottenburg, the amount of data the routing system has to %work with is manageable and can be modeled with the basic Dijkstra implementation.

\subsection{Location detection}
The main system used for outdoor localization in Google Maps and other similar services is GPS \cite{google_maps}. During usage of the navigation app, the user gets the opportunity to activate it and give permission for its usage. While Google's fused location API is the default location provider for Android devices \cite{fused_location_api}, IOS devices use Apple's Core Location framework \cite{core_location_framework}.

Although GPS is the standard system for localization, several other techniques are often used to improve pedestrian location detection on mobile devices. One relevant for this thesis is pedestrian dead reckoning.

Pedestrian dead reckoning localizes the user by tracking its steps and the movement direction. In a mobile scenario, this can be accomplished with the built-in device sensors. Steps can be detected with a distinct pattern in the retrieved acceleration, while the heading angle is read from the magnetometer \cite{PDR}. Additionally, the system needs a reference point for localization: Since only the distance of movement in a certain direction is tracked, the starting point has to be known. Another crucial factor for correct PDR is a precise estimation of step length \cite{PDR} and a correct calibration of used sensors. The following formula presents a standard way to calculate the current location in a 2-dimensional environment with step-count $i$, position $(x, y)$, estimated step length $L$ and heading $\phi$ \cite{PDR_2}.

\begin{equation}
    \begin{pmatrix}x_{i}\\y_{i}\end{pmatrix} = \begin{pmatrix}x_{i-1}\\y_{i-1}\end{pmatrix} + L * \begin{pmatrix}sin(\phi_{i-1})\\cos(\phi_{i-1})\end{pmatrix}
\end{equation}

One of the biggest disadvantages of PDR is the fact that errors accumulate during usage. Imprecisions in magnetometer, acceleration and step length values can result in incorrect PDR localization for long-distance usage.

\cite{localization_techniques} describes an approach for merging data of the smartphone's built-in sensors for PDR with GPS functionality using a Kalman filter. It tries to use pedestrian dead reckoning as a base for localization and corrects its error accumulation with additional position lookup through GPS. The work states that such a technique can improve localization accuracy as well as energy-saving in a pedestrian localization scenario.

\subsection{Estimated time of arrival (ETA)}
Providing an estimation of the time it takes to overcome a suggested route is an important task for modern navigation systems. Google Maps calculates the estimated time of arrival (ETA) based on several different factors \cite{google_maps}, including:

\begin{itemize}
    \item Distance from start point to destination
    \item Average speed of the user and its selected form of transportation
    \item Current traffic situation on the chosen route
    \item Official and recommended speed limits
    \item Road types
    \item Historical average speed data
    \item Collected data from users who traveled similar routes
\end{itemize}

By collecting and merging this information, Google Maps manages it to provide its users with a precise ETA \cite{google_maps}. Nevertheless, since no concrete formula is specified for such a calculation, a simpler ETA algorithm that utilizes a subset of named parameters is implemented in this thesis.

\subsection{Visual design}
The next step after retrieving the user's location and calculating the best route with its respective ETA is the presentation of the results within the app. \cite{visual_design_of_navigation_system} provides an overview of different solutions for information presentation within a mobile navigation system. It compares auditory instructions with route presentation of a map, simple route presentation on a map, display of current location and direction as well as a list of textual descriptions.

After evaluating all four different methods, \cite{visual_design_of_navigation_system} suggests the following aspects for a successful presentation of navigation data:

\begin{itemize}
    \item A map oriented in the current walking direction should show the user's current location as well as the taken route and information about the map (landmarks, street names)
    \item The map should provide a zoom function
    \item Textual instructions are the most inefficient way of communicating navigation information
    \item Audio instruction can improve the map functionality, although GPS accuracy has to be taken into account when providing information about distance
\end{itemize}

\section{Location-based services}
Location-based services are mobile applications/services, which utilize the ability to localize the user to provide customized information \cite{location_based_services}. By offering its users information about nearby restaurants, supermarkets, shops, hotels and other POI, mobile navigation systems such as Google \cite{google_maps_website} or Apple Maps \cite{apple_maps_website} also fall into the category of such services. Other popular apps using location-based systems are emergency services, tour guides, delivery tracking systems, social-media platforms with location-sharing functionalities, weather apps and mobility or taxi apps \cite{geofencing_and_background_tracking} \cite{Sadhukhan2021}.

Location-based mobile apps nowadays usually use the in-build GPS capabilities of mobile devices to provide their services in outdoor environments. Techniques used for indoor environments, on the other hand, are WiFI, RFID, ZigBee and Bluetooth \cite{Sadhukhan2021}.

Since the core functionality of the software developed in this thesis is comparable to mobile navigation systems such as Google or Apple Maps, the goal of this thesis itself can be described as a location-based service.

\subsection{Geofencing}
Geofencing describes a class of techniques used to automatically send notifications based on entering or leaving a specified geographic location \cite{geofencing}. These geographic areas are usually specified geometrically, by placing circles, polygons or polylines into specific locations, or symbolically by specifying a location by name such as a state, a country or an address \cite{geofencing}.

There are several use cases for geofencing including marketing systems that send promotional notifications when users enter defined areas, logistic systems, that track and detect when certain deliveries and vehicles enter destination locations, mobile apps for navigation and social networking, as well as safety applications for tracking and detecting dangerous areas. One of the most prominent examples of geofence usage in a mobile application for marketing purposes is Burger King's "Whopper detour" campaign launched in 2019 \cite{burger_king_whopper_detour}. The company sold a Whopper for one cent to every user who placed an order via the Burger King app within a McDonald's restaurant.

One key enabler for geofencing is an extensive background tracking functionality \cite{geofencing_and_background_tracking}. In the case of modern mobile devices, smartphone apps with a geofencing functionality need the ability to constantly query the user's location, even if the app is in the background.

%\section{Collection of publicly available data from the web}