\chapter{Introduction}
\label{cha:introduction}
\section{Context and background}
TU Berlin's main campus in Charlottenburg is one of the biggest continuous campuses in Europe. It consists of more than 60 different buildings, is home to over 35000 students and has an area of over 600000 square meters. While studying is an important reason to visit the campus, learning and attending courses are not the only tasks that take place on it. Campus Charlottenburg provides TU Berlin's members the opportunity to connect to other like-minded people, eat meals in one of the cafeterias or bakeries, as well as to participate in one of the many social events that take place on it. It is also the workplace for over 7000 different professors, students, teachers and researchers and can be therefore considered as an important place in the lives of TU Berlin's members.

\section{Motivation}
TU Berlin's enormous size and complexity come with several difficulties for its members. One main challenge in this context is the difficulty of navigation and orientation on the campus. Especially new students of TU Berlin, who are not profound in the campus structure, often have a hard time localizing unknown buildings. Obtaining campus-specific information is also complicated: Described navigation difficulties are topped by an overwhelming amount of information about and around the campus, which is mainly distributed across several different web platforms and parties.

A working, manageable and modern digital information culture is one of the most significant key factors for overcoming those difficulties and for successful study, work and life on campus Charlottenburg. Web resources about the campus should be thereby presented in an easy and intuitive way and the strengths of software should be utilized, to provide more accessible methods for orientation and navigation on the campus.

Looking at the current state of the digital information culture of TU Berlin, several websites provided by the university and Studierendenwerk Berlin, as well as a manual map for brief orientation on the campus, can be seen as the most important sources of information for TU Berlin's members. Established mobile navigation providers can be furthermore also listed as tools, that are commonly used to conquer the described orientation problems. This landscape of web resources and platforms nevertheless comes with a set of problems:

One of them is the fact that there is no fitting digital solution for navigation on the campus. Students currently have two options when choosing a tool to navigate across it: On the one hand, TU Berlin provides a downloadable image of campus Charlottenburg, which gives a basic aerial overview of the whole campus. Although this solution is straightforward and understandable, it only provides the possibility to navigate manually over the campus. It also does not take full advantage of the potential, that comes with a digital solution.

On the other hand, there are several external digital navigation system providers such as Google or Apple Maps. Those systems take advantage of digital tooling and provide a modern way of navigation. The main problem with these systems is the fact that they do not contain a detailed mapping of campus Charlottenburg. Searching for several buildings is difficult/impossible and since these systems are not profound for most of the paths on campus, suggested routes are often based on public streets around TU Berlin's main area rather than the faster ones inside of it.

Another problem is the fact that the amount and size of provided platforms is, in itself, overwhelming and complicated. There is no central entity that provides all relevant data about campus Charlottenburg. Instead, the main actors in this context, TU Berlin and Studierendenwerk Berlin, provide different web pages and platforms for buildings, courses, meals, rooms, events, job markets and much more. This complex, often confusing, separation of data harms the usability of the system and therefore the overall digital information culture. A central entity, containing all information, would be a beneficial addition to the system.

The fact that most of the information is provided in the form of websites has another consequence for members of TU Berlin: A fast and easy information lookup on mobile devices, one of the most present computer architectures across the target audience, is often slow, unresponsive and unnecessarily complicated.

\section{Proposed solution}
This bachelor's thesis tries to investigate the current digital information culture at TU Berlin and wants to solve information retrieval and navigation difficulties by providing an implementation of a digital information and navigation system for mobile devices. Both of these features use an underlying offline map of campus Charlottenburg, which therefore serves as a base layer for the app's main screen. The following sections provide a more detailed overview of the described key features:

Campus map: A custom map of campus Charlottenburg represents the main element of the app. It comes completely bundled with the application for offline usage and displays the most relevant information about important entities on the campus, e.g., buildings, cafeterias, green areas and pathways. The map is designed in a way to provide its users with an easy and intuitive way for orientation on the campus, with a high geographical resolution and immediately recognizable entities.

Navigation system: A fast, reliable and offline-usable navigation system overlays the digital map and helps users find their way across the campus. It contains all the geodata and points of interest (POI) of the campus and can reliably calculate the fastest route between them. It also integrates the mobile device's GPS capabilities to use the current user location during live navigation.

Information layer: The second layer on the map displays information about campus Charlottenburg. It fetches and parses them automatically from publicly available web resources and maps the information onto their respective POI. Examples of such mappings can be the portrayal of meal plans for canteens on campus, timetable overviews for specific rooms as well as the emphasis of events on the map.