\chapter*{Zusammenfassung}
\label{cha:zusammenfassung}

Der Campus ist der wichtigste Ort für das soziale Zusammenleben, Studieren und Arbeiten aller Angehörigen einer Universität. Er beherbergt, je nach Größe der Universität, eine Vielzahl von Einrichtungen, Instituten und Fakultäten, sodass es einer guten Orientierungsfähigkeit der Studierenden und Lehrenden bedarf. Ein entscheidender Faktor für ein funktionierendes Leben am Campus ist eine durchdachte und gut nutzbare digitale Informationskultur der entsprechenden Universität.

Mit einem der größten zusammenhängenden Campus in Europa stellt die Technische Universität (TU) Berlin ihre Angehörigen regelmäßig vor Herausforderungen: Schwierigkeiten beim Orientieren und eine komplexe, über mehrere Webseiten verteilte Informationskultur erschweren die Übersicht und das Leben am Campus.

Diese Arbeit versucht dem entgegenzuwirken, indem sie die Implementation einer Campus-App für mobile Geräte aufzeigt, welche Angehörigen der TU Berlin mittels digitaler Karte die Möglichkeit bietet, sich auf dem Campus einfacher zurechtzufinden. Unterstützt wird diese digitale Karte von einer Informationsschicht, welche versucht, alle wichtigen Informationen des Campus gebündelt darzustellen. Die App wird anschließend quantitativ mit den populärsten mobilen Navigationssystemen hinsichtlich der Auflösung des Campus und der Qualität vorgeschlagener Routen verglichen. Dies resultiert in einer Verdoppelung (Google Maps) / Verfünffachung (Apple Maps) der geografischen Abdeckung des Campus, einer durchschnittlichen Zeitersparnis von jeweils 30 \% pro Route (Google und Apple Maps) und einer durchschnittlichen Distanzersparnis von 23 \% (Google Maps) und 20 \% (Apple Maps).
